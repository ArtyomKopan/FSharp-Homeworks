\documentclass{article}
\usepackage[utf8]{inputenc}
\usepackage[T2A]{fontenc}
\usepackage{mathtools}
\usepackage{amsmath}
\usepackage{amssymb}
\usepackage{hyperref}

\title{Homework3}
\author{Artyom Kopan}

\begin{document}
    \huge
    \maketitle

    \section{Task 3.1}
    \begin{equation}
        ((\lambda a.(\lambda b.b\:b)(\lambda b.b\:b))\:b)((\lambda c.(c\:b))(\lambda a.a))
    \end{equation}
    $\beta$-конверсия
    \begin{equation}
        (\lambda a.(\lambda b.b\:b)(\lambda b.b\:b))\:b)\:((\lambda a.a)\:b)
    \end{equation}
    $\beta$-конверсия
    \begin{equation}
        ((\lambda a.(\lambda b.b\:b)(\lambda b.b\:b))\:b)\:b
    \end{equation}
    $\beta$-конверсия
    \begin{equation}
    (\lambda b.b\:b)\:(\lambda b.b\:b)\:b
    \end{equation}

    Это выражение не является нормальной формой, так как редукции можно продолжать. Но если продолжать применение $\beta$-конверсии, то терм не изменится. Однако утверждается, что редукция по нормальной стратегии является полной в том смысле, что если терм имеет нормальную форму, то редукция по нормальной стратегии в конце концов достигнет её (см. \href{https://en.wikipedia.org/wiki/Beta\_normal\_form}{Beta\_normal\_form}).
    Мы получили противоречие, значит, данный терм не имеет нормальной формы.

\end{document}